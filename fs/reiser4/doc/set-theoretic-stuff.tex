\documentclass[a4paper, oneside, fleqn]{article}

\usepackage{latexsym}
\usepackage{url}
\usepackage[T2A]{fontenc}

\pagestyle{empty}
\listfiles
\setcounter{errorcontextlines}{100}
\makeindex
\pagestyle{headings}
\frenchspacing
\tolerance=1000
\parindent=0pt
\raggedbottom
\setlength\parskip{6pt}

\DeclareMathAlphabet{\mathbsf}{T2A}{cmss}{b}{n}
\SetMathAlphabet{\mathbsf}{normal}{T2A}{cmss}{b}{n}

\def\qopname@#1{\mathop{\fam 0#1}\nolimits}
\newcommand{\mathsign}[1]
	{\index{#1@$\mathbsf{#1}$}\qopname@{\mathbsf{#1}}}

\def\As{\mathsign{Assoc}}
\newcommand{\svi}[2]
    {\texttt{[} #1 \ V \texttt{]}}

\begin{document}

\thispagestyle{empty}

%\section{Definitions}

We have a set $X$ of objects, and ``associated-with'' relation. We shall write

$$a\As b, \quad a\in X, \ b\in X$$

to denote that $a$ is associated with $b$.

One can imagine $\As$ relation as graph where elements of $X$ are nodes and
where there is arc (arrow) from $a$ to $b$ iff $a$ is associated with
$b$. Note that no further restrictions are placed on $\As$. In particular, it
is not supposed that $\As$ is reflexive (object is not necessary associated
with itself), symmetric, or transitive.

$\beta(X)$ is set of all subsets of $X$, that is $$\beta(X) = \{ U \subseteq X
\}$$

Let's define function $A:X\to^{}\beta(X)$ as follows:

$$A(x)=\{y\in X\ |\ y\As x\}, \quad x\in X.$$

that is $A(x)$ is a set of all objects in $X$ associated with $x$.
Then, define \mbox{$A^*:\beta(X)\to^{}\beta(X)$} as follows:

$$A^*(U)=\bigcup\limits_{x\in U} A(x), \quad U\subseteq X.$$

that is, $A(U)$ is set of all objects associated with any element of $U$. Now
we can define $\svi{U}{V}$, where $U, V\subseteq X$---``set vicinity
intersection'' operation as:

%\begin{displaymath}
%A^+(U) = \left\{
%    \begin{array}{rl}
%    U = \{x\}      & \Rightarrow A(x),\\
%    \textrm{else}  & \Rightarrow A^*(U)
%    \end{array} \right.
%\end{displaymath}

$$\svi{U}{V} = A^*(U) \cap A^*(V).$$

In other words, $\svi{U}{V}$ is a set of all objects associated with some
element of $U$ \emph{and} some element of $V$.

\end{document}

% Local variables:
% indent-tabs-mode: nil
% tab-width: 4
% eval: (progn (if (fboundp 'flyspell-mode) (flyspell-mode)) (set (make-local-variable 'compile-command) "latex set-theoretic-stuff.tex ; dvips -o set-theoretic-stuff.ps set-theoretic-stuff.dvi"))
% End:
